%
% GNU courseware, XIN YUAN, 2017
%

\section{群聚智能}

\frame{
\centerline{\textbf{\Huge{群聚智能}}}
}

\frame{\frametitle{定义}

无智能或简单智能的主体通过任何形式的聚集协作而表现出智能行为的特性。

~

群,是一组相互之间可以进行直接或间接通信(通过改变局部环境)的主体。
}

\frame{\frametitle{思想}

源于分子自动机系统的自组织研究。
在没有集中且不提供全局模型的前提下,为寻找复杂分布式问题解决方案提供基础。
}

\frame{\frametitle{思想}
相对于传统的梯度优化算法

~

	\begin{itemize}
		\item<1-> 完全分布式处理个体间和个体环境间作用,具有自组织性;
		\item<2-> 个体间非直接,合作基于环境感知,系统可扩展和安全;
		\item<3-> 无集中控制,具有鲁棒性,个别个体失效不会影响整个问题求解;
		\item<4-> 个体智能简单,实现方便,执行时间短。
	\end{itemize}
}

\frame{\frametitle{种类}
	\begin{itemize}
		\item<1-> 蚁群(蚁狮)
		\item<2-> 鸟群(粒子群)
		\item<3-> 人工鱼群
		\item<4-> 蜂群
		\item<5-> 蛙群
	\end{itemize}
}

\frame{\frametitle{种类}
	\begin{itemize}
		\item<1-> 萤火虫
		\item<2-> 蝙蝠群
		\item<3-> 布谷鸟
		\item<4-> 麻雀
		\item<5-> 社会性蜘蛛
	\end{itemize}
}

\frame{\frametitle{种类}
	\begin{itemize}
		\item<1-> 灰狼
		\item<2-> 猫群
		\item<3-> 狮群
		\item<4-> 斑点鬣狗
		\item<5-> 鲸鱼捕食
		\item<6-> 磷虾
		\item<7-> 帝企鹅
	\end{itemize}
}

\frame{\frametitle{种类}
	\begin{itemize}
		\item<1-> 蟑螂
		\item<2-> 蝗虫
		\item<3-> 蝴蝶
		\item<4-> 飞蛾扑火
		\item<5-> 大红斑蝶
		\item<6-> 天牛须
	\end{itemize}
}

\frame{\frametitle{种类}
	\begin{itemize}
		\item<1-> 黏菌
		\item<2-> 蚯蚓
		\item<3-> 大象放牧
		\item<4-> 樽海鞘群
		\item<5-> 乌燕鸥
		\item<6-> 共生生物搜索
	\end{itemize}
}

\frame{\frametitle{种类}
	\begin{itemize}
		\item<1-> 花朵授粉
		\item<2-> 烟花
	\end{itemize}
}

\frame{\frametitle{公式示例}

\begin{equation*}
p_{ij}^{k}(t) = \left\{
	\begin{array}{lc}
	\frac{[\tau_{ij}(t)]^{\alpha}\cdot[\eta_{ij}(t)]^{\beta}}{\sum\limits_{s\in J_{k}(i)} [\tau_{is}(t)]^{\alpha}\cdot[\eta_{is}(t)]^{\beta}}, & \mbox{如果} J \in J_{k}(i) \\
	0, & \mbox{否则}
	\end{array} \right.
\end{equation*}
}

\frame{\frametitle{定理示例}

\newtheorem{zh-thm}{定理}
\begin{zh-thm}
如果有a,b,c, 则有$a^2+b^2=c^2$。
\end{zh-thm}

\renewcommand\proofname{证明}
\begin{proof}
因为有a,b,c, 所以有$a^2+b^2=c^2$。
\end{proof}
}

\frame{\frametitle{算法示例}

\begin{algorithm}[H]
\caption{计算 $y=x^n$}\label{alg1}
\algsetup{linenosize=\tiny} \scriptsize
	\begin{algorithmic}
		\REQUIRE $n \geq 0 \vee x \neq 0$
		\ENSURE $y=x^n$
		\STATE $y \Leftarrow 1$
		\IF{$n < 0$}
			\STATE $X \Leftarrow 1 / x$
			\STATE $N \Leftarrow -n$
		\ELSE
			\STATE $X \Leftarrow x$
			\STATE $N \Leftarrow n$
		\ENDIF
		\WHILE{$N \neq 0$}
			\IF{$N$ is even}
				\STATE $X \Leftarrow X \times X$
				\STATE $N \Leftarrow N / 2$
			\ELSE[$N$ is odd]
				\STATE $y \Leftarrow y \times X$
				\STATE $N \Leftarrow N - 1$
			\ENDIF
		\ENDWHILE
	\end{algorithmic}
\end{algorithm}
}

\frame{\frametitle{流程图示例}

\tikzstyle{block}=[rectangle,draw,fill=blue!20,text width=4em,text centered,rounded corners]
\tikzstyle{huge-block}=[rectangle,draw,fill=cyan!20,text width=5em,text centered,rounded corners,minimum height=4em]
\tikzstyle{line}=[draw]
\begin{tikzpicture}[node distance=2cm, auto]
\path[use as bounding box] (-4, 0) rectangle (10, -2);
\path[line]<1-> node[huge-block](center){群聚智能};
\path[line, <->]<2-> node[block, below of=center, node distance=3cm](center-develop){蚁群算法}
(center)--(center-develop);
\path[line, <->]<3-> node[block, left of=center-develop, node distance=3cm](left-develop){粒子群}
(center)--(left-develop);
\path[line, <->]<4-> node[block, right of=center-develop, node distance=3cm](right-develop){灰狼}
(center)--(right-develop);
\end{tikzpicture}
}

\frame{\frametitle{流程图示例}

\begin{tikzpicture}[font={\sf \scriptsize}]

\node (start) at (-2,3) [draw, terminal, minimum width=2cm, minimum height=0.5cm] {开始};
\node (getdata) at (-2,2) [draw, predproc, align=left, minimum width=2cm, minimum height=0.5cm] {读取数据};
\node (init) at (-2,1) [draw, process, minimum width=3cm, minimum height=0.5cm] {设置参数和初始化};
\coordinate (point1) at (-0.25,1);
\node (eval) at (1,1) [draw, process, minimum width=2cm, minimum height=0.5cm] {评价蚁群};
\node (decide) at (1,-1) [draw, decision, minimum width=2cm, minimum height=0.5cm] {判断终止条件};
\node (storage) at (1,-3) [draw, storage, minimum width=3cm, minimum height=0.5cm] {输出结果并存储};
\node (iter) at (4,0) [draw, process, minimum width=3cm, minimum height=0.5cm] {迭代次数增加};
\node (select) at (4,0.7) [draw, process, minimum width=3cm, minimum height=0.5cm] {概率选择移动方向};
\node (update) at (4,1.4) [draw, process, minimum width=3cm, minimum height=0.5cm] {信息素更新};
\coordinate (point2) at (4,2);
\node (end) at (4,-3) [draw, terminal, minimum width=2cm, minimum height=0.5cm] {结束};

\draw[->] (start) -- (getdata);
\draw[->] (getdata) -- (init);
\draw[->] (init) -- (point1) -- (eval);
\draw[->] (eval) -- (decide);
\draw[->] (decide) -| node[right]{否} (iter);
\draw[->] (decide) -- node[right]{是} (storage);
\draw[->] (iter) -- (select);
\draw[->] (select) -- (update);
\draw[->] (update) -- (point2) -| (point1);
\draw[->] (storage) -- (end);
\end{tikzpicture}
}

\frame{\frametitle{动画示例}
\begin{figure}[hb]
\centering
\animategraphics[height=2.5in, autoplay, controls]{1}{topics/swarm-files/lzq_}{0}{5}
\end{figure}
}

%end
