%
% GNU courseware, XIN YUAN, 2018
%

\documentclass[11pt]{beamer}
\usepackage{beamerthemesplit}
%url
\usepackage{hyperref}
\hypersetup{colorlinks=true, urlcolor=blue}
%xeCJK
\usepackage[BoldFont,SlantFont,CJKchecksingle]{xeCJK}
\setCJKmainfont[BoldFont=SimHei,SlantedFont=KaiTi]{SimSun}
\setCJKsansfont[BoldFont=SimHei,SlantedFont=KaiTi]{SimSun}
\setCJKmonofont[ItalicFont={Adobe Fangsong Std}]{SimSun}
\setCJKfamilyfont{zhsong}{SimSun}
%\setCJKfamilyfont{zhhei}{Adobe Song Std}
\setCJKfamilyfont{zhhei}{SimHei}
%\setCJKfamilyfont{zhhei}{Adobe Heiti Std}
\setCJKfamilyfont{zhkai}{KaiTi}
\setCJKfamilyfont{zhfs}{FangSong}
\usetheme{goettingen}
\usecolortheme{beetle}

\parindent 2em

%title
\title{生物智能与算法}
\author{袁昕}
\date{\today}

\begin{document}

\frame{\titlepage}

\frame{
\centerline{\textbf{\Huge{基础知识}}}
}

%tables

\section*{大纲}

\frame{\tableofcontents}

%summary

\section{基础}

\frame{\frametitle{科普}
	\begin{itemize}
		\item<1-> 科学网,果壳,科普中国等
		\item<2-> 图形和视频摘要
		\item<3-> 折纸
	\end{itemize}
}

\frame{\frametitle{计算智能的早期发展}
	\begin{itemize}
		\item<1-> 自动控制的故事
		\item<2-> 系统辨识(带有激励的数据实验)
		\item<3-> 工业大数据是否有效的辨析
		\item<4-> 一般智能:从模式识别到人工智能
	\end{itemize}
}

\frame{\frametitle{自动控制的知识体系}
	\begin{itemize}
		\item<1-> 自动化学科:参见\href{https://blog.sciencenet.cn/blog-3387802-1331861.html}{网页}
		\item<2-> 图示:参见\href{https://blog.sciencenet.cn/blog-3387802-1331871.html}{网页}
		\item<3-> 基本问题:参见\href{https://blog.sciencenet.cn/blog-3387802-1331872.html}{网页}
		\item<4-> 多智能体和复杂网络:参见\href{https://blog.sciencenet.cn/blog-3387802-1380777.html}{网页}
		\item<5-> 理论和工程:参见\href{https://blog.sciencenet.cn/blog-3316223-1368152.html}{网页}
	\end{itemize}
}

%optimization

\section{最优化理论}

\frame{\frametitle{最优化}
	\begin{itemize}
		\item<1-> 问题描述
		\item<2-> 求解方法
		\item<3-> 正则化的本质(约束是什么,假设有噪声引起偏离)
		\item<4-> 泛函角度
		\item<5-> 物理角度
		\item<6-> 几何直观角度(微分几何)
		\item<7-> 统计角度(频率、贝叶斯、公理)
	\end{itemize}
}

\frame{\frametitle{问题描述}
	\begin{itemize}
		\item<1->凸函数和非凸函数
		\item<2->线性函数和非线性函数
		\item<3->有限维空间和无限维空间
		\item<4->离散函数(如组合优化)和连续函数(能量泛函及最大似然)
	\end{itemize}
}

\frame{\frametitle{拉格朗日乘子法}
几何直观解释:

~

参见\href{http://blog.sciencenet.cn/blog-3428464-1283235.html}{网页}
}

\frame{\frametitle{正则化}
几何直观解释:

~

参见\href{https://mp.weixin.qq.com/s?__biz=MzU0MDQ1NjAzNg==&mid=2247541375&idx=1&sn=2c4e882d69e0cbec156eec71a38cf599&chksm=fb3a8774cc4d0e62899bb78208be93d1bd2fe3798090c00c9a9ae078a2cd3bd98a6444594cb4&scene=27}{网页}
}

\frame{\frametitle{求解}
	\begin{itemize}
		\item<1-> 传统的数值优化算法。
		\item<2-> 现代的生物智能优化算法。
	\end{itemize}
}

\frame{\frametitle{求解-传统}
传统的数值优化算法,以严格的数学原理为指导,通过多次迭代,不断逼近全局最优解的算法,
比如基于梯度的算法、几何规划算法和二次规划算法等等。
这类算法的求解过程可以简述如下:首先,依据经验选定一个初始方案,
然后依据严格的数学原理来对初始方案进行不断的迭代,达到最大迭代次数,算法就停止了,
所获取的最终方案,即为找到的可能的全局最优方案。显然,这类算法对于初始方案的依赖性非常强。
鉴于数学原理推导的缜密性,一旦初始方案选定之后,最终方案实际上已经确定了。
因此,如果初始方案选择的较为合适,会得到一个比较理想的解决方案;否则,则会得到一个糟糕的解决方案。
此外,这类算法通常还会需要梯度信息,会大大增加求解的复杂程度。
}

\frame{\frametitle{求解-现代}
现代的生物智能优化算法,是通过随机性和简单规则的结合,通过模拟自然现象,来实现对优化问题的求解,
比如经典的粒子群优化算法和差分进化算法等等。
这类算法的求解过程可以简述如下:首先,对种群随机初始化,然后依据设定的简单规则来引导种群的搜索方向,
达到最大迭代次数了,算法就停止了,所获取的最终方案即为找到的可能的全局最优方案。
显然,这类算法由于在搜索机制中引入了随机数以及种群的随机初始化,就大大弱化了这类算法对于初始方案依赖程度。
此外,这类算法不需要梯度信息。
}

\frame{\frametitle{求解-对比}
在求解复杂优化问题的时候,相较于传统的数值优化算法,生物智能优化算法展现出更强的避免局部最优解的能力。
当前,生物智能优化算法已经作为一种非常主流的优化技术广泛应用不同领域的优化问题。
这一结论目前已经被发表的大量论文所证实。
}

%end

\frame{\frametitle{结束}
\centerline{\textbf{\Huge{结束!}}}
}

\frame{\frametitle{版权申明}

本作品采用知识共享 署名-非商业性使用-禁止演绎 3.0 中国大陆 许可协议进行许可。
要查看该许可协议,可访问 http://creativecommons.org/licenses/by-nc-nd/3.0/cn/ 或者
写信到 Creative Commons, PO Box 1866, Mountain View, CA 94042, USA。
}

\end{document}
