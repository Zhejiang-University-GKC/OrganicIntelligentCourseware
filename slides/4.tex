%
% GNU courseware, XIN YUAN, 2023
%

\documentclass[11pt]{beamer}
\usepackage{beamerthemesplit}
%url
\usepackage{hyperref}
\hypersetup{colorlinks=true, urlcolor=blue}
%xeCJK
\usepackage[BoldFont,SlantFont,CJKchecksingle]{xeCJK}
\setCJKmainfont[BoldFont=SimHei,SlantedFont=KaiTi]{SimSun}
\setCJKsansfont[BoldFont=SimHei,SlantedFont=KaiTi]{SimSun}
\setCJKmonofont[ItalicFont={Adobe Fangsong Std}]{SimSun}
\setCJKfamilyfont{zhsong}{SimSun}
%\setCJKfamilyfont{zhhei}{Adobe Song Std}
\setCJKfamilyfont{zhhei}{SimHei}
%\setCJKfamilyfont{zhhei}{Adobe Heiti Std}
\setCJKfamilyfont{zhkai}{KaiTi}
\setCJKfamilyfont{zhfs}{FangSong}
\usetheme{goettingen}
\usecolortheme{beetle}

\parindent 2em

%title
\title{生物智能与算法}
\author{袁昕}
\date{\today}

\begin{document}

\frame{\titlepage}

\frame{
\centerline{\textbf{\Huge{理论提升}}}
}

%tables

\section*{大纲}

\frame{\tableofcontents}

%physics

\section{数学与物理}

\frame{\frametitle{数学和物理}
物理理论和数学描述之间的关系:
参见\href{http://blog.sciencenet.cn/blog-39419-1271041.html}{网页}

~

全局尺度与局部尺度:
参见\href{http://blog.sciencenet.cn/blog-39419-1273157.html}{网页}
}

\frame{\frametitle{物理的历史发展和思维}
	\begin{itemize}
	\item<1-> 从牛顿力学到相对论:参见\href{https://mp.weixin.qq.com/s/amhuQYFELB8gmhr2d5XgMg}{网页}
	\item<2-> 深刻理解侠义相对论:参见\href{https://blog.sciencenet.cn/blog-3377-1325424.html}{网页}
	\item<3-> 郎道理解狭义相对论:参见\href{https://blog.sciencenet.cn/blog-3377-1326910.html}{网页}
	\item<4-> 能量守恒:参见\href{http://blog.sciencenet.cn/blog-39346-1293732.html}{网页}
	\item<5-> 经典力学:参见\href{https://blog.sciencenet.cn/blog-709494-1353458.html}{网页}
	\end{itemize}
}

%books

\section{数学教材}

\frame{\frametitle{教科书}
	\begin{itemize}
		\item<1-> 微积分、集合到拓扑
		\item<2-> 集合、测度、概率论、统计到实分析
		\item<3-> 平面几何、解析几何、线性代数到向量代数
		\item<4-> 初等代数、复杂微积分、一维到多维面积计算、复数
		\item<5-> 复数、向量空间、超越空间、射影几何、无穷到有限、黎曼球面到复分析
		\item<6-> 有限和无限、傅里叶级数、泛函分析、傅里叶变换、对偶空间、组合数学到最优化理论
	\end{itemize}
}

\frame{\frametitle{教科书}
	\begin{itemize}
		\item<1-> 函数映射的一对多和多对一,空间的扭曲回环,非线性,空间嵌入,到升维和降维处理。
		\item<2-> 高阶函数、张量代数,流形,到泛函。
		\item<3-> 随机过程、随机微分方程、组合数学、离散数学、运筹学到博弈论。
		\item<4-> 数理金融、数学工具与实验观察不符、质疑和倒查、用平稳研究的非平稳,随机变量和样本轨道可能搞混。
		\item<5-> 抽象代数、微分几何到代数几何。
	\end{itemize}
}

%mathematics

\section{提升数学水平}

\frame{\frametitle{数学体系}
\begin{enumerate}
\item<1-> 数学体系,参见\href{https://mp.weixin.qq.com/s/Jo7ID5xl8eOOFvUcGh8Umw}{网页}
\item<2-> 数理逻辑两种方法要结合,参见\href{https://blog.sciencenet.cn/blog-692836-1373598.html}{网页}
\end{enumerate}
}

\frame{\frametitle{如何学好高数}
如何学好高数

~

参见\href{https://mp.weixin.qq.com/s/a2wpf7sj80IXSQwEbs_S9A}{网页}
}

\frame{\frametitle{数学即数数或数圈}
\begin{enumerate}
\item<1-> 数得准即代数
\item<2-> 数不准即分析(几何辅助)
\item<3-> 数不清即几何拓扑(无穷小,无穷大,说不清,概率统计)
\item<4-> 怎么数即逻辑
\end{enumerate}
}

\frame{\frametitle{关于线性代数}
线性代数的矩阵(对比解析几何;处理坐标系无关的量)

~

参见\href{https://mp.weixin.qq.com/s/DB6_w-TuaFCxR6Fws8l99w}{网页}
}

\frame{\frametitle{关于统计}
\begin{enumerate}
\item<1-> 统计学思想总结,参见\href{https://mp.weixin.qq.com/s/Shbhd2SaAg2DmikrDdOuSw}{网页}
\item<2-> 概率的意义:随机世界与大数法则,参见\href{https://mp.weixin.qq.com/s/yxyaEcMolGMen3-65nWCdg}{网页}
\item<3-> 测量误差与统计推断问题的统一性,参见\href{https://blog.sciencenet.cn/blog-103568-1346732.html}{网页}
\end{enumerate}
}

\frame{\frametitle{关于几何}
微分几何

~

参见\href{https://mp.weixin.qq.com/s/nk1ta8jVrqubwhSHekq0tA}{网页}
}

\frame{\frametitle{不确定性}
为何要融合Mixture?参见\href{http://blog.sciencenet.cn/blog-388372-1299920.html}{网页}

~

两大载体:变量、模型。
}

\frame{\frametitle{不确定性}
\begin{enumerate}
\item<1-> 现实世界不可知变量的不确定性往往用noise建模,如系统噪声、量测噪声等。
\item<2-> 不确定的模型或者假设往往混合mixture形式来表达,如高斯混合GMM,多模型MM,交互多模型IMM,
多假设跟踪器MHT,蒙特卡洛方法(粒子群混合),随机有限集统计学框架下的多伯努利混合MBM,广义标签多伯努利GLMB,等。
还有Watson mixture,inverted Beta mixture,von-Mises Fisher mixture model,
以及复合型混合:Gaussian-Student's-t mixture,Gaussian-uniform mixture等。
\end{enumerate}
}

\frame{\frametitle{不确定性}
思想是多模型/多分布加权平均(Arithmetic average),即AA融合模型,
也与神经网络neuron network(NN)有着内在关联(NN可看作大量神经元的mixture混合工作机制)。
可以进一步从几何角度思考。
}

\frame{\frametitle{应用数学}
应用数学的强大威力

~

参见\href{https://mp.weixin.qq.com/s/ks05jYJHG-bleVcPhDWQhA}{网页}
}

\frame{\frametitle{探讨}
\begin{enumerate}
\item<1-> 微分和导数,是线性化的开端,结合了无穷的思想,
参见\href{http://blog.sciencenet.cn/blog-3377-1289944.html}{网页}
\item<2-> 概率定义的探讨,相对频率、个人主观信念度、倾向度。
参见\href{https://blog.sciencenet.cn/blog-3503579-1321302.html}{网页}
\item<3-> 随机过程,参见\href{http://blog.sciencenet.cn/home.php?mod=space&uid=3418723&do=blog&id=1286356}{网页1}
\href{https://blog.sciencenet.cn/blog-3418723-1365894.html}{网页2}
\end{enumerate}
}

%philosophy

\section{科学哲学}

\frame{\frametitle{经典科学哲学思路}
	\begin{itemize}
		\item<1-> 还原论
		\item<2-> 因果论(穆勒五法)
		\item<3-> 逻辑学(同一律,排中律,矛盾律,因果律,区分对立和矛盾的概念)
		\item<4-> 本体论
		\item<5-> 科学研究的范式(数据驱动,开普勒范式;基本原理驱动,牛顿范式)
	\end{itemize}
}

\frame{\frametitle{现代科学哲学思路}
	\begin{itemize}
		\item<1-> 整体论
		\item<2-> 系统论(结合控制论,讨论所有可能的变量和规律,调整事物之间的关系)
		\item<3-> 层次论(从高阶函数、泛函升级)
		\item<4-> 相关性和因果性(数据驱动和原理驱动,因果是幻觉吗?)
		\item<5-> 互为因果和循环论证(宗教的神逻辑)
	\end{itemize}
}

\frame{\frametitle{科学哲学}
一流的科学家对哲学也应有深刻的认识,参见\href{http://blog.sciencenet.cn/blog-3387120-1273402.html}{网页}

~

库恩科学哲学范式理论,参见\href{http://blog.sciencenet.cn/blog-71079-1273375.html}{网页}

~

波普尔的理论,参见\href{http://blog.sciencenet.cn/blog-1255140-769490.html}{网页}
}

\frame{\frametitle{科学难题}
科学难题的形成

~

参见\href{https://blog.sciencenet.cn/blog-39419-1272556.html}{网页}
}

%end

\frame{\frametitle{结束}
\centerline{\textbf{\Huge{结束!}}}
}

\frame{\frametitle{版权申明}

本作品采用知识共享 署名-非商业性使用-禁止演绎 3.0 中国大陆 许可协议进行许可。
要查看该许可协议,可访问 http://creativecommons.org/licenses/by-nc-nd/3.0/cn/ 或者
写信到 Creative Commons, PO Box 1866, Mountain View, CA 94042, USA。
}

\end{document}
